\chapter{Aufgabenanalyse}
\label{chap:Aufgabenanalyse}
\section{Analyse}

\vspace{-0.3cm}
\section{Eingabeformat}

\clearpage
\vspace{-0.3cm}
\section{Ausgabeformat}

\vspace{-0.3cm}
\section{Anforderung an das Programm}

\vspace{-0.3cm}
\section{Sonderfälle}
\label{sec:Sonderfaelle}

\vspace{-0.3cm}
\section{Fehlerfälle}
\label{sec:Fehlerfaelle}
Die auftretenden Fehler kann man in drei verschiedene Fehlerarten aufteilen. Zu diesen gehören technische, syntaktische und semantische Fehlerfälle.\newline Technische Fehler liegen vor, wenn die angegebene Datei nicht vorhanden ist, durch fehlende Zugriffsrechte nicht gelesen werden kann, oder die Ausgabedatei durch fehlende Schreibrechte nicht erstellt werden kann. Syntaktische Fehler treten auf, wenn die Eingabedatei nicht dem vorgegebenen Format entspricht, bei semantischen Fehlern sind die Daten in der Eingabedatei fehlerhaft.

Durch die Analyse der Aufgabenstellung und des Eingabeformated ergeben sich folgende syntaktische Fehlerfälle:
\begin{itemize}
    \item Mögliche Fehlerfälle auflisten
\end{itemize}

Durch die Analyse der Aufgabenstellung und des Eingabeformates ergeben sich folgende semantische Fehlerfälle:
\begin{itemize}
	\item Mögliche Fehlerfälle auflisten
\end{itemize}

Die Behandlung der Fehlerfälle wird in der Verfahrensbeschreibung angegeben.
\cleardoublepage