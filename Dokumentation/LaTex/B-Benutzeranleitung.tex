\chapter{Benutzeranleitung}
\label{Benutzeranleitung}
\vspace{-0.2cm}
\section{Verzeichnisstruktur der Abgabe}
Im Prüfungsprodukt sind folgende Dateien und Verzeichnisse vorhanden:
\begin{description}
	\item [Wurzelverzeichnis]
\end{description}		
	\begin{itemize}
		\item vorkompilierte Version des Programms
		\item Skript zum automatischen Ausführen des Programms mit mehreren Eingabedateien
		\item Skript zum Kompilieren des Programms
		\item Skript zum Löschen aller erstellten Ausgabedateien
	\end{itemize}
\begin{description}[\setlabelphantom{Output}]
	\item [Output] Enthält die Ausgabedateien des Programms.
	\item [src] Enthält den Quellcode des Programms.
	\item [Tests] Enthält beispielhafte Eingabedateien.
	\item [Doku] Enthält die Dokumentation.
\end{description}
\vspace{-0.2cm}
\section{Vorbereitung des Systems}
\subsection{Systemvoraussetzungen}
Um das Programm verwenden zu können, wird ein Microsoft Betriebssystem in der Version 8 oder höher benötigt. Auf dem Betriebssystem muss die Windows PowerShell sowie eine Java Runtime Environment (JRE) der Version 8 oder höher installiert sein.

Die PowerShell sollte nach Angaben von Microsoft auf jedem Windows Rechner mit einem Betriebssystem von Windows 8 oder höher standardmäßig installiert sein.

Sollte die JRE nicht oder in einer veralteten Version installiert sein, kann dies unter der folgenden Adresse heruntergeladen werden:
\newline
\url{https://www.java.com/de/download/}

\subsubsection*{PowerShell einrichten}
Die PowerShell untersagt das Ausführen von Skripten in der Standardeinstellung. Damit dies geändert werden kann, muss die PowerShell als Administrator gestartet werden.

Um die PowerShell als Administrator zu starten, geht man in den Startmenüeintrag der PowerShell. Diesen finden Sie durch das öffnen des Startmenüs und anschließender Eingabe von \glqq PowerShell\grqq Dort machen Sie einen Rechtsklick auf \glqq Windows PowerShell\grqq , und klicken in dem aufkommendem Menü auf den Punkt \glqq Als Administrator ausführen\grqq. Je nach Sicherheitseinstellungen des Betriebssystems erscheint ein Fenster mit der Nachfrage, ob Änderungen durch das Programm zugelassen werden sollen, dies wird mit \glqq Ja\grqq\ bestätigt. Gegebenenfalls müssen Sie auch Ihr Administratorpasswort eingeben um fortfahren zu können. Anschließend öffnet sich dann die Windows PowerShell mit Administratorrechten.

Als erstes sollten die aktuellen Einstellungen mittels \lstinline{Get-ExecutionPolicy} ausgelesen und notiert werden, damit die Einstellungen später wieder zurückgesetzt werden können. Im nächsten Schritt werden die Richtlinien geändert. Dies geschieht über den Befehl \lstinline{Set-ExecutionPolicy RemoteSigned}. \lstinline{RemoteSigned} bedeutet, dass Skripte, die aus dem Internet heruntergeladen wurden, signiert sein müssen um ausgeführt zu werden. Lokal erstellte Skripte werden immer ausgeführt.

Falls die Skripte nicht ausgeführt werden sollten, müssen die Richtlinien weiter herabgesetzt werden, dies geschieht über \lstinline{Set-ExecutionPolicy Unrestricted}. \lstinline{Unrestricted} bedeutet, dass alle Skripte ausgeführt werden, jedoch wird für unsignierte Skripte, die aus dem Internet stammen, eine Warnung ausgegeben.

Die Richtlinien der PowerShell können in den ursprünglichen Zustand mittels des Befehls \lstinline{Set-ExecutionPolicy <POLICY>} zurückgesetzt werden. Dabei ist \lstinline{<POLICY>} der im ersten Schritt ausgelesene Wert. Um die Voreinstellungen der PowerShell zu erhalten, wird als Parameter \lstinline{Restricted} angegeben.


\subsection{Installation}
Der gesamte Inhalt der Abgabe ist in ein beliebiges und beschreibbares Verzeichnis zu kopieren. Danach ist das Programm betriebsbereit.

\clearpage
\section{Kompilieren}
Ein vorkompiliertes und ausführbares Programm liegt in Form von \lstinline{GroProLDahlberg.jar} der Abgabe bei. Soll das Programm wegen Änderungen im Quelltext, oder aus anderen Gründen neu kompiliert werden, kann mit dem Skript \lstinline{compile.ps1} das Programm neu kompiliert werden.

Damit dies erfolgreich ausgeführt werden kann, darf die vorliegende Verzeichnisstruktur nicht verändert werden, weil das Skript speziell auf diese Struktur angepasst ist. Weiter ist erforderlich, dass ein Compiler für Java (JDK) der Version 8 in der Umgebungsvariable \lstinline{PATH} eingebunden ist. Ist der Compiler nicht in der Umgebungsvariable funktioniert der Compilerbefehl nicht.

Um die \lstinline{PATH}-Variable zu erweitern, geht man im Startmenü mit einem Rechtsklick auf \glqq Computer\grqq $\rightarrow$ \glqq Eigenschaften\grqq. Dort wird \glqq Erweiterte Systemeinstellungen\grqq\ am linken Rand ausgewählt. Auf der Registerkarte \glqq Erweitert\grqq\ wird der unterste Knopf \glqq Umgebungsvariablen...\grqq\ gewählt. Im nächsten Schritt wählt man die \lstinline{PATH}-Variable aus, dabei ist es egal, ob nun die Benutzer- oder die Systemvariable ändert. Jedoch werden zur Änderung der Systemvariable Administratorrechte benötigt. Nach der Auswahl der \lstinline{PATH}-Variable wird auf den Knopf \glqq Bearbeiten...\grqq\ gedrückt. Im Feld für den Wert der Variable wird am Ende folgendes angehängt:

\lstinline{;<Pfad zum JDK bin Ordner>}\newline
Falls ein JDK einer höheren Version vorhanden ist, kann auch dieses verwendet werden. Hier wird allerdings nicht garantiert, dass der Kompiliervorgang erfolgreich ist.

Beim anfügen in die \lstinline{PATH}-Variable ist es wichtig auf das Semikolon zu achten, da dieses als Trennzeichen agiert.

Wenn diese Bedingungen erfüllt sind, kann das PowerShell-Skript \lstinline{compile.ps1} ausgeführt werden. Dabei wird automatisch die Datei \lstinline{GroProLDahlberg.jar} neu erstellt.

\clearpage
\section{Programmaufruf}
\label{sec:Programmaufruf}
Um einzelne Dateien zu verarbeiten, müssen diese dem Programm \lstinline{GroProLDahlberg.jar} über die Kommandozeile oder PowerShell als Parameter mitgegeben werden. In beiden Fällen erstellt das Programm eine Ausgabedatei, die den ursprünglichen Dateinamen um \glqq .out\grqq\ erweitert.

\section{Testen der Beispiele}
Sollen mehrere Testbeispiele verarbeitet werden, müssen diese in einem Verzeichnis zusammen gefasst werden. Durch das PowerShell-Skript {start.ps1} werden diese dann nacheinander verarbeitet. Das Skript hat als optionale Parameter \lstinline{DIR} und \lstinline{TYP}. Über \lstinline{DIR} kann ein Verzeichnis mit den Testfällen angegeben werden und über \lstinline{TYP} die Endung der Dateien. Die default-Werte sind für \lstinline{DIR ./Tests} und für \lstinline{TYP .txt}. Diese werden verwendet, wenn keine anderen Werte übergeben werden. Ein Beispiel für den Aufruf ist:

\lstinline{./start.ps1 TYP=.input}
\cleardoublepage